% Lecture notes for Physical Mathematics I
% Ewan Stewart
% KAIST
% 2011/11/25

\documentclass[a4paper,12pt]{report}

\usepackage{pm1}

\setcounter{chapter}{3}
\setcounter{section}{1}
\setcounter{page}{14}

\begin{document}

\section{Linear operators}

\keyword{Linear operators} are linear mappings from the Hilbert space to itself.

\subsection{Hermitian conjugate, inverse and commutator}

The \keyword{Hermitian conjugate} $A^\dagger$ of an operator $A$ is defined by
\begin{equation}
\bra\phi A^\dagger \ket\psi = \left( \bra\psi A \ket\phi \right)^*
\end{equation}
and the \keyword{inverse} $A^{-1}$ of an operator $A$ is defined by
\begin{equation}
A^{-1} A = A A^{-1} = 1
\end{equation}
where $1$ is the identity operator.

The \keyword{commutator} of two operators $A$ and $B$ is given by
\begin{equation}
[A,B] = A B - B A
\end{equation}
They are said to commute if $[A,B] = 0$.

\subsection{Hermitian, unitary and projection operators}

A \keyword{Hermitian operator} $H$ has the property
\begin{equation}
H^\dagger = H
\end{equation}
and a \keyword{unitary operator} $U$ has the property
\begin{equation}
U^\dagger = U^{-1}
\end{equation}
A unitary transformation
\begin{eqnarray}
\ket\phi & \to & U \ket\phi
\\
\bra\psi & \to & \bra\psi U^\dagger
\\
A & \to & U A U^\dagger
\end{eqnarray}
leaves contractions invariant
\begin{equation}
\bra\psi A \ket\phi \to \bra\psi A \ket\phi
\end{equation}

A \keyword{projection operator} $P$ is a Hermitian operator with the property
\begin{equation}
P^2 = P
\end{equation}
It projects onto the subspace $\hilbert{P} = \{ \ket\phi : P \ket\phi = \ket\phi \}$
and annihilates its orthogonal complement $\hilbert{P}^\perp$
\begin{equation}
P \ket\phi =
\left\{
\begin{array}{ccl}
\ket\phi & \textrm{for} & \ket\phi \in \hilbert{P}
\\
0 & \textrm{for} & \ket\phi \in \hilbert{P}^\perp
\end{array}
\right.
\end{equation}
For example
\begin{equation}
P = \frac{ \ket\phi \bra\phi }{ \braket\phi\phi }
\end{equation}
and $1 - P$ are both projection operators.

\subsection{Eigenspaces}

An \keyword{eigenvector} $\ket\phi$ of an operator $A$ satisfies
\begin{equation} \label{eigenvector}
A \ket\phi = \alpha \ket\phi
\end{equation}
where the \keyword{eigenvalue} $\alpha$ is a scalar.
Any linear combination of eigenvectors with the same eigenvalue $\alpha$ is also an eigenvector with eigenvalue $\alpha$, so eigenvectors with the same eigenvalue form a subspace of the Hilbert space called an \keyword{eigenspace}
\begin{equation}
\hilbert{A}_\alpha = \left\{ \ket\phi : A \ket\phi = \alpha \ket\phi \right\}
\end{equation}
The information that \eq{eigenvector} gives us about $A$ can be distilled as
\begin{equation} \label{eseq}
A P_{\hilbert{A}_\alpha} = \alpha P_{\hilbert{A}_\alpha}
\end{equation}
where $P_{\hilbert{A}_\alpha}$ is the projection operator that projects onto the eigenspace $\hilbert{A}_\alpha$.

If $A$ is a Hermitian or unitary operator
\footnote{Or more generally a normal operator: $[N,N^\dagger]=0$.}
then its eigenspaces $\hilbert{A}_\alpha$ are \keyword{orthogonal}
\begin{equation}
P_{\hilbert{A}_\alpha} P_{\hilbert{A}_\beta} = 0
\qquad \left( \alpha \neq \beta \right)
\end{equation}
and \keyword{complete}
\begin{equation}
\sum_\alpha P_{\hilbert{A}_\alpha} = 1
\end{equation}
Therefore, using \eq{eseq}, $A$ can be decomposed as
\begin{equation}
A = \sum_\alpha \alpha P_{\hilbert{A}_\alpha}
\end{equation}
See Figure~\ref{fig:eigenspace}.
\begin{figure}
\centering
\begin{tikzpicture}
\draw[red] (-3,0) -- (3,0) node[right]{$\hilbert{A}_{-1}$};
\draw[blue] (0,-1) -- (0,3) node[above]{$\hilbert{A}_{2}$};
\draw[->] (0,0) -- (1,1) node[above right]{$\ket\phi$};
\draw[dotted] (1,1) -- (1,0);
\draw[dotted] (1,1) -- (0,1);
\draw[->] (0,0) -- (-1,2) node[above left]{$A \ket\phi$};
\draw[dotted] (-1,2) -- (-1,0);
\draw[dotted] (-1,2) -- (0,2);
\end{tikzpicture}
\caption{ \label{fig:eigenspace}
$A \ket\phi = - P_{\hilbert{A}_{-1}} \ket\phi + 2 P_{\hilbert{A}_{2}} \ket\phi$.
}
\end{figure}

Let $A$ and $B$ be Hermitian or unitary operators with eigenspaces $\hilbert{A}_\alpha$ and $\hilbert{B}_\beta$ respectively.
Then the intersections of their eigenspaces $\hilbert{A}_\alpha \cap \hilbert{B}_\beta$ are orthogonal
\begin{equation}
P_{\hilbert{A}_\alpha \cap \hilbert{B}_\beta} P_{\hilbert{A}_\gamma \cap \hilbert{B}_\delta} = 0
\qquad \left( \alpha,\beta \neq \gamma,\delta \right)
\end{equation}
The intersections are complete if and only if $A$ and $B$ commute
\begin{equation}
\sum_{\alpha,\beta} P_{\hilbert{A}_\alpha \cap \hilbert{B}_\beta} = 1
\quad \Leftrightarrow \quad
[A,B] = 0
\end{equation}
Thus if $A$ and $B$ commute we can decompose them in terms of a common set of projection operators
\begin{eqnarray}
A & = & \sum_{\alpha,\beta} \alpha P_{\hilbert{A}_\alpha \cap \hilbert{B}_\beta}
\\
B & = & \sum_{\alpha,\beta} \beta P_{\hilbert{A}_\alpha \cap \hilbert{B}_\beta}
\end{eqnarray}
Note that
\begin{equation}
P_{\hilbert{A}_\alpha \cap \hilbert{B}_\beta} = P_{\hilbert{A}_\alpha} P_{\hilbert{B}_\beta}
\quad \Leftrightarrow \quad
\left[ P_{\hilbert{A}_\alpha} , P_{\hilbert{B}_\beta} \right] = 0
\quad \Leftrightarrow \quad
\left[ A , B \right] = 0
\end{equation}

\end{document}


