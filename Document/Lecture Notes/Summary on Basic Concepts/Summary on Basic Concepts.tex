% Copyright 2004 by Till Tantau <tantau@users.sourceforge.net>.
%
% In principle, this file can be redistributed and/or modified under
% the terms of the GNU Public License, version 2.
%
% However, this file is supposed to be a template to be modified
% for your own needs. For this reason, if you use this file as a
% template and not specifically distribute it as part of a another
% package/program, I grant the extra permission to freely copy and
% modify this file as you see fit and even to delete this copyright
% notice. 

\documentclass{beamer}

% There are many different themes available for Beamer. A comprehensive
% list with examples is given here:
% http://deic.uab.es/~iblanes/beamer_gallery/index_by_theme.html
% You can uncomment the themes below if you would like to use a different
% one:
%\usetheme{AnnArbor}
%\usetheme{Antibes}
%\usetheme{Bergen}
%\usetheme{Berkeley}
%\usetheme{Berlin}
%\usetheme{Boadilla}
%\usetheme{boxes}
%\usetheme{CambridgeUS}
%\usetheme{Copenhagen}
%\usetheme{Darmstadt}
%\usetheme{default}
%\usetheme{Frankfurt}
%\usetheme{Goettingen}
%\usetheme{Hannover}
%\usetheme{Ilmenau}
%\usetheme{JuanLesPins}
%\usetheme{Luebeck}
\usetheme{Madrid}
%\usetheme{Malmoe}
%\usetheme{Marburg}
%\usetheme{Montpellier}
%\usetheme{PaloAlto}
%\usetheme{Pittsburgh}
%\usetheme{Rochester}
%\usetheme{Singapore}
%\usetheme{Szeged}
%\usetheme{Warsaw}

\usepackage{kotex}
\usepackage{braket}
\usepackage{array}
\usepackage{calc}
\usepackage{datetime}


\graphicspath{{images/}}
\usepackage{listings}

\usepackage{kotex} 
\usepackage{hyperref}
\usepackage{listings}
\usepackage{color}

\definecolor{mygreen}{rgb}{0,0.6,0}
\definecolor{mygray}{rgb}{0.5,0.5,0.5}
\definecolor{mymauve}{rgb}{0.58,0,0.82}

\lstset{ 
  backgroundcolor=\color{white},   % choose the background color; you must add \usepackage{color} or \usepackage{xcolor}; should come as last argument
  basicstyle=\footnotesize,        % the size of the fonts that are used for the code
  breakatwhitespace=false,         % sets if automatic breaks should only happen at whitespace
  breaklines=true,                 % sets automatic line breaking
  captionpos=b,                    % sets the caption-position to bottom
  commentstyle=\color{mygreen},    % comment style
  deletekeywords={...},            % if you want to delete keywords from the given language
  escapeinside={\%*}{*)},          % if you want to add LaTeX within your code
  extendedchars=true,              % lets you use non-ASCII characters; for 8-bits encodings only, does not work with UTF-8
  frame=single,	                   % adds a frame around the code
  keepspaces=true,                 % keeps spaces in text, useful for keeping indentation of code (possibly needs columns=flexible)
  keywordstyle=\color{blue},       % keyword style
  language=Python,                 % the language of the code
  morekeywords={*,...},            % if you want to add more keywords to the set
  numbers=left,                    % where to put the line-numbers; possible values are (none, left, right)
  numbersep=5pt,                   % how far the line-numbers are from the code
  numberstyle=\tiny\color{mygray}, % the style that is used for the line-numbers
  rulecolor=\color{black},         % if not set, the frame-color may be changed on line-breaks within not-black text (e.g. comments (green here))
  showspaces=false,                % show spaces everywhere adding particular underscores; it overrides 'showstringspaces'
  showstringspaces=false,          % underline spaces within strings only
  showtabs=false,                  % show tabs within strings adding particular underscores
  stepnumber=2,                    % the step between two line-numbers. If it's 1, each line will be numbered
  stringstyle=\color{mymauve},     % string literal style
  tabsize=2,	                   % sets default tabsize to 2 spaces
  title=\lstname                   % show the filename of files included with \lstinputlisting; also try caption instead of title
}


\title{Introduction to Python and Programming Language}

% A subtitle is optional and this may be deleted
\subtitle{Koscom Algorithm Lecture}

\author{신승우}
% - Give the names in the same order as the appear in the paper.
% - Use the \inst{?} command only if the authors have different
%   affiliation.

% \institute[Universities of Somewhere and Elsewhere] % (optional, but mostly needed)
% {
  % \inst{1}%
  % Department of Computer Science\\
  % University of Somewhere
  % \and
  % \inst{2}%
  % Department of Theoretical Philosophy\\
  % University of Elsewhere}
% - Use the \inst command only if there are several affiliations.
% - Keep it simple, no one is interested in your street address.

% - Either use conference name or its abbreviation.
% - Not really informative to the audience, more for people (including
%   yourself) who are reading the slides online

\subject{Theoretical Computer Science}

% This is only inserted into the PDF information catalog. Can be left
% out. 

% If you have a file called "university-logo-filename.xxx", where xxx
% is a graphic format that can be processed by latex or pdflatex,
% resp., then you can add a logo as follows:

% \pgfdeclareimage[height=0.5cm]{university-logo}{university-logo-filename}
% \logo{\pgfuseimage{university-logo}}

% Delete this, if you do not want the table of contents to pop up at
% the beginning of each section:


% \AtBeginSection[]
% {
  % \begin{frame}<beamer>{Outline}
    % \tableofcontents[currentsection,hideallsections]
  % \end{frame}
% }

% Let's get started
\begin{document}

\begin{frame}
  \titlepage
\end{frame}

\begin{frame}{Outline}
  \tableofcontents
  % You might wish to add the option [pausesections]
\end{frame}

% Section and sections will appear in the presentation overview
% and table of contents.

% \begin{frame}[fragile]{hello.py}
% 이제는 위에서 했던 것과 거의 비슷한 것을 할 텐데, 다만 다른 방식으로 해 볼 것입니다. 화면에서 보이는 대로 따라하세요. 
% \begin{lstlisting}[language=Python]
% print('Hello World!')
% \end{lstlisting}
% \end{frame}

\section{Computation and Computation Model} 


\begin{frame}{Motivating Example} 

소인수분해 알고리즘 
\begin{itemize} 
\item 일반 컴퓨터 : O($n^k$)
\item 양자 컴퓨터 : O($log^k n$)
\end{itemize}

왜 이런 차이가 날까요?
\end{frame}

\section{Concept of Computation}

\begin{frame}{Concept of Computation}
Computation이란? 

\begin{itemize} 
\item \textbf{Initial Setting}에서 시작해서 
\item \textbf{정해진 operation}들을 유한 번 거쳐 
\item Output을 결정
\end{itemize}

하는 과정을 Computation이라고 한다. 여기서 Initial Setting과 정해진 operation들은 어떻게 정해질까요? 이것은 Computational Model에 의해서 정해집니다. 
\end{frame}

\subsection{Computation Model} 

\begin{frame}{Computation Models} 

Computation Model은 정의하기에 따라 매우 다양한데, 보통 많이 쓰이는 모델들은 다음과 같습니다. 

\begin{itemize} 
\item RAM Model 
\item Turing Machine 
\item Decision Trees
\end{itemize}

우리가 일반적으로 생각하는 프로그래밍 언어들은 대부분 Turing Machine과 같은 계산력을 가집니다. 즉, 튜링머신으로 풀 수 있는 문제는 일반 프로그래밍 언어로 풀 수 있고, 그 반대도 성립합니다. 
\end{frame} 


\subsection{Computability} 


\begin{frame}{Halting Problem}

파이썬 언어로 되어있는 어떤 임의의 함수 f에 대해서, 이 함수가 멈추는지 알려주는 함수를 파이썬으로 짤 수 있을까?

\end{frame}

\subsection{Accepting Subroutines : Concept of Oracle}

\begin{frame}{Concept of Oracle} 

Computation Model은 종종 오라클이라는 외부 계산기기를 가지기도 합니다. 이 때, computation model은 기존의 operation에 더해서, 이 오라클에 input을 넣고 output을 받는 operation을 할 수 있습니다. 이 오라클 내에서의 계산과정은 black box이며, 원 computation model에서는 전혀 알 수 없습니다. 

이는 프로그래밍에서 subroutine의 개념과 정확히 일치합니다. 이러한 subroutine 중 많이 쓰이는 것들을 모아놓은 것이 Abstract Data Type입니다. 

\end{frame}

\section{ADT vs Data Structure} 

\subsection{Abstraction}

\begin{frame}{Motivating Example}
순서에 대해서 생각해 보자. 
\begin{itemize} 
\item Total Order : 정수 간의 순서 / 문자열 간의 순서 / ... 
\item Partial Order : 대학교 과목 간의 순서 / 할 일의 순서 / ... 
\end{itemize}
\end{frame}

\begin{frame}{Abstraction} 

개별적인 문제에서, 공통된 부분을 추출하는 것! 

\end{frame}

\begin{frame}{Abstract Data Type}

많이 쓰이는 \textbf{operation들}을 모아놓은 것. 예를 들면, 

\begin{itemize} 
\item 선입선출되는 상황에서 많이 쓰이는 operation set : Queue
\item 선입후출되는 상황에서 많이 쓰이는 operation set : Stack
\item 계층구조 등을 나타낼 때 많이 쓰이는 operation set : Tree
\end{itemize}

로 생각할 수 있습니다. 여기서, operation들만 모아놓은 것이지, 어떤 식으로 그 operation을 구현할지는 전혀 언급되지 않았습니다. 
\end{frame}

\subsection{Data Structure} 

\begin{frame}{Interface vs Implementation} 
위와 같은 operation set들은 함수들 간의 일종의 인터페이스를 제공합니다. 위 오라클 머신에서, 오라클의 계산 과정은 전혀 밝혀지지 않았음을 상기하세요. 우리가 아는 것은 함수들의 집합일 뿐입니다. 자바에서는 이와 같은 개념을 interface라 하며, C++에서는 Abstract class라고 합니다. 파이썬에서는 abc 모듈을 이용하여 나타낼 것입니다. 

현실에서 이 인터페이스를 쓰기 위해서는 이를 구현해야 합니다. 이를 어떻게 구현할지 모아 놓은 것을 데이터구조라고 합니다. 
\end{frame}

\begin{frame}{Data Structure}

데이터구조의 예시에는 다음과 같은 것들이 있습니다. 

\begin{itemize} 
\item Array 
\item Linked List : Doubly Linked List, Cyclic Linked List, ...
\item Binary Tree 
\item Heap : Binary Heap, Fibonacci Heap, ... 
\end{itemize}

\end{frame}

\begin{frame}{Example : Priority Queue}
Priority Queue를 구현하기 위해서, 다음의 두 데이터구조를 생각해 봅시다. 

\begin{itemize} 
\item Linked List 
\item Sorted Linked List : 새 원소가 들어올 때, 리스트가 언제나 정렬된 상태를 유지하도록 함. 
\end{itemize}

이 때, Priority Queue의 operation들의 복잡도는 어떻게 될까요?
\end{frame}

\section{Recursion} 

\subsection{Analyzing Recursion} 

\begin{frame}{Recursive Function Example : Fibonacci Sequence}

실습 : 피보나치 수열은 f(n) = f(n-1) + f(n-2)를 만족하는 수열을 말합니다. 이 때, 피보나치 수를 계산하는 함수를 짜 보면 어떻게 될까요? 
\end{frame}


\begin{frame}{Computational Complexity : Fibonacci Sequence}

이 때, 위에서 짠 알고리즘의 복잡도는 어떻게 계산할 수 있을까요? 

\end{frame}

\subsection{Dynamic Programming}

\begin{frame}{Fibonacci Sequence Revisited}

실습 : 위에서 짠 알고리즘이 복잡한 이유는 같은 것을 여러 번 계산하기 때문입니다. 따라서, 이를 개선하기 위해서 기존에 계산한 것을 저장하는 리스트를 만들고, 그 리스트를 업데이트하면서 계산해 봅시다. 
\end{frame}

\begin{frame}{Fibonacci Sequence Revisited} 

추가 실습 : 더 빠른 방법은 없을까요? 행렬을 사용해 보세요. 

\end{frame}


\end{document}


