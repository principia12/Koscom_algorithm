% Copyright 2004 by Till Tantau <tantau@users.sourceforge.net>.
%
% In principle, this file can be redistributed and/or modified under
% the terms of the GNU Public License, version 2.
%
% However, this file is supposed to be a template to be modified
% for your own needs. For this reason, if you use this file as a
% template and not specifically distribute it as part of a another
% package/program, I grant the extra permission to freely copy and
% modify this file as you see fit and even to delete this copyright
% notice. 

\documentclass{beamer}

% There are many different themes available for Beamer. A comprehensive
% list with examples is given here:
% http://deic.uab.es/~iblanes/beamer_gallery/index_by_theme.html
% You can uncomment the themes below if you would like to use a different
% one:
%\usetheme{AnnArbor}
%\usetheme{Antibes}
%\usetheme{Bergen}
%\usetheme{Berkeley}
%\usetheme{Berlin}
%\usetheme{Boadilla}
%\usetheme{boxes}
%\usetheme{CambridgeUS}
%\usetheme{Copenhagen}
%\usetheme{Darmstadt}
%\usetheme{default}
%\usetheme{Frankfurt}
%\usetheme{Goettingen}
%\usetheme{Hannover}
%\usetheme{Ilmenau}
%\usetheme{JuanLesPins}
%\usetheme{Luebeck}
\usetheme{Madrid}
%\usetheme{Malmoe}
%\usetheme{Marburg}
%\usetheme{Montpellier}
%\usetheme{PaloAlto}
%\usetheme{Pittsburgh}
%\usetheme{Rochester}
%\usetheme{Singapore}
%\usetheme{Szeged}
%\usetheme{Warsaw}

\usepackage{kotex}
\usepackage{braket}
\usepackage{array}
\usepackage{calc}
\usepackage{datetime}


\graphicspath{{images/}}
\usepackage{listings}

\usepackage{kotex} 
\usepackage{hyperref}
\usepackage{listings}
\usepackage{color}

\definecolor{dkgreen}{rgb}{0,0.6,0}
\definecolor{gray}{rgb}{0.5,0.5,0.5}
\definecolor{mauve}{rgb}{0.58,0,0.82}

\definecolor{mygreen}{rgb}{0,0.6,0}
\definecolor{mygray}{rgb}{0.5,0.5,0.5}
\definecolor{mymauve}{rgb}{0.58,0,0.82}

\lstset{ 
  backgroundcolor=\color{white},   % choose the background color; you must add \usepackage{color} or \usepackage{xcolor}; should come as last argument
  basicstyle=\footnotesize,        % the size of the fonts that are used for the code
  breakatwhitespace=false,         % sets if automatic breaks should only happen at whitespace
  breaklines=true,                 % sets automatic line breaking
  captionpos=b,                    % sets the caption-position to bottom
  commentstyle=\color{mygreen},    % comment style
  deletekeywords={...},            % if you want to delete keywords from the given language
  escapeinside={\%*}{*)},          % if you want to add LaTeX within your code
  extendedchars=true,              % lets you use non-ASCII characters; for 8-bits encodings only, does not work with UTF-8
  frame=single,	                   % adds a frame around the code
  keepspaces=true,                 % keeps spaces in text, useful for keeping indentation of code (possibly needs columns=flexible)
  keywordstyle=\color{blue},       % keyword style
  language=Python,                 % the language of the code
  morekeywords={*,...},            % if you want to add more keywords to the set
  numbers=left,                    % where to put the line-numbers; possible values are (none, left, right)
  numbersep=5pt,                   % how far the line-numbers are from the code
  numberstyle=\tiny\color{mygray}, % the style that is used for the line-numbers
  rulecolor=\color{black},         % if not set, the frame-color may be changed on line-breaks within not-black text (e.g. comments (green here))
  showspaces=false,                % show spaces everywhere adding particular underscores; it overrides 'showstringspaces'
  showstringspaces=false,          % underline spaces within strings only
  showtabs=false,                  % show tabs within strings adding particular underscores
  stepnumber=2,                    % the step between two line-numbers. If it's 1, each line will be numbered
  stringstyle=\color{mymauve},     % string literal style
  tabsize=2,	                   % sets default tabsize to 2 spaces
  title=\lstname                   % show the filename of files included with \lstinputlisting; also try caption instead of title
}


\lstdefinestyle{python}{frame=tb,
  language=Python,
  aboveskip=3mm,
  belowskip=3mm,
  showstringspaces=false,
  columns=flexible,
  basicstyle={\small\ttfamily},
  numbers=left,
  numberstyle=\tiny\color{gray},
  keywordstyle=\color{blue},
  commentstyle=\color{dkgreen},
  stringstyle=\color{mauve},
  breaklines=true,
  breakatwhitespace=true,
  tabsize=4
}


\title{Introduction to Python and Programming Language}

% A subtitle is optional and this may be deleted
\subtitle{Koscom Algorithm Lecture}

\author{신승우}
% - Give the names in the same order as the appear in the paper.
% - Use the \inst{?} command only if the authors have different
%   affiliation.

% \institute[Universities of Somewhere and Elsewhere] % (optional, but mostly needed)
% {
  % \inst{1}%
  % Department of Computer Science\\
  % University of Somewhere
  % \and
  % \inst{2}%
  % Department of Theoretical Philosophy\\
  % University of Elsewhere}
% - Use the \inst command only if there are several affiliations.
% - Keep it simple, no one is interested in your street address.

% - Either use conference name or its abbreviation.
% - Not really informative to the audience, more for people (including
%   yourself) who are reading the slides online

\subject{Theoretical Computer Science}

% This is only inserted into the PDF information catalog. Can be left
% out. 

% If you have a file called "university-logo-filename.xxx", where xxx
% is a graphic format that can be processed by latex or pdflatex,
% resp., then you can add a logo as follows:

% \pgfdeclareimage[height=0.5cm]{university-logo}{university-logo-filename}
% \logo{\pgfuseimage{university-logo}}

% Delete this, if you do not want the table of contents to pop up at
% the beginning of each section:


% \AtBeginSection[]
% {
  % \begin{frame}[fragile]<beamer>{Outline}
    % \tableofcontents[currentsection,hideallsections]
  % \end{frame}
% }

% Let's get started
\begin{document}

\begin{frame}
  \titlepage
\end{frame}

\begin{frame}{Outline}
  \tableofcontents
  % You might wish to add the option [pausesections]
\end{frame}

% Section and sections will appear in the presentation overview
% and table of contents.

% \begin{frame}[fragile][fragile]{hello.py}
% 이제는 위에서 했던 것과 거의 비슷한 것을 할 텐데, 다만 다른 방식으로 해 볼 것입니다. 화면에서 보이는 대로 따라하세요. 
% \begin{lstlisting}[language=Python]
% print('Hello World!')
% \end{lstlisting}
% \end{frame}

\section{Basic Syntax of the Python Language} 

\subsection{Basic Syntax} 

\begin{frame}[fragile]{Hello World!}
\begin{lstlisting}[language=Python]
print('Hello World!')
\end{lstlisting}

세미콜론이 없음에 주의해주세요. C나 자바와는 다르게 세미콜론이 없습니다. 
\end{frame}

\begin{frame}{Indentation}
파이썬 언어는 다른 언어에서의 괄호를 indent로 대신합니다. 표준적으로, 인덴트는 space 4개로 하며, 이를 위해서 대부분의 개발 도구에서는 tab을 space 4개로 대체하여 사용합니다. 만약 인터넷이나 다른 곳에서 소스를 사용할 때 indentation 관련 에러가 난다면, tab/space 여부를 체크해보세요. 

\end{frame}

\begin{frame}[fragile]{= vs == vs is}
\begin{lstlisting}[language=Python]
a = 1
a == 1 
a is 1 # True or False?
\end{lstlisting}

\begin{itemize} 
\item = : assignment 
\item == : equality check
\item is : address check 
\end{itemize}
\end{frame}

\begin{frame}[fragile]{if-elif-else}
if-elif-else 구문에서는 항상 indent를 유지해야 합니다. 
\begin{lstlisting}[language=Python]
if cond1:
    .... # when cond1 is True
elif cond2:    
    .... # when cond1 is False and cond2 is True 
else:
    .... # when cond1 and cond2 is False
\end{lstlisting}

\end{frame}


\begin{frame}[fragile]{while}
\begin{lstlisting}[language=Python]
while(cond):
    ....  # while cond remains True
\end{lstlisting}

다른 대부분의 언어들과 비슷하게, break문을 통해서 벗어날 수 있습니다. 
\end{frame}

\begin{frame}[fragile]{for loop}
\begin{lstlisting}[language=Python]
for i in range(10):
    ....
\end{lstlisting}

문법은 for *1 in *2: 형태를 취하며, 이 때 
\begin{itemize}
\item *1 변수에 *2에 있는 원소를 \textit{차례로} 대입하며, 
\item indent로 명시된 코드블럭 내 구문을 수행합니다. 
\end{itemize}

즉, *2에 있는 부분은 순서가 있는 자료형이여야 합니다. 이에 대해서는 조금 후에 더 자세하게 다루겠습니다. 

\end{frame}

\begin{frame}[fragile]{switch}
파이썬에서는 switch문을 지원하지 않습니다. 하지만, 비슷하게 만드는 방법이 있습니다. 이에 대해서는 파이썬 사전을 배운 후 다루겠습니다. 
\end{frame}

\begin{frame}[fragile]{assert}
\begin{lstlisting}[language=Python]
assert False
assert cond, ...
\end{lstlisting}
\end{frame}

\subsection{Primitive Types and Supported Methods} 

\begin{frame}[fragile]{Boolean}

많은 다른 언어에서 나오는 Boolean과 같으며, 다음의 operation들을 지원합니다. 
\lstinputlisting[style=python, 
                caption={Boolean Example  (example\_bool1.py)}]{"../../Reference/examples/example_bool1.py"}

\end{frame}

\begin{frame}[fragile]{Numeric}
\lstinputlisting[style=python, 
                caption={Numeric Types  (example\_numeric1.py)}]{"../../Reference/examples/example_numeric1.py"}
\end{frame}

\begin{frame}[fragile]{list}
list 데이터형은 다음의 연산들을 지원합니다. 
\begin{itemize} 
\item a in d : 배열(d) 안에 특정 원소(a)가 있는지를 검사합니다. 
\item + : 배열 두 개를 이어서 새로운 배열을 만듭니다. 같은 데이터형이여야 합니다. 
\item d[i] : d의 i번째 원소를 반환합니다. 
\item d[i:j:k] : d의 i번째 원소부터 j번째 원소까지, k번째 원소마다 선택하여 리스트를 만들어 반환합니다. 
\item d.index(elem) : d에서 elem 이 처음으로 나오는 위치를 반환합니다. 
\item d.append(elem) : elem을 d 맨 뒤에 추가합니다. 
\end{itemize}

\end{frame}
\begin{frame}[fragile]{list}
\lstinputlisting[style=python, 
                caption={Operations for Sequential Types  (example\_sequence2.py)}]{"../../Reference/examples/example_sequence2.py"}

\end{frame}

\begin{frame}[fragile]{list vs iterator}
\end{frame}

% \begin{frame}[fragile]{String}
% \end{frame}

\begin{frame}[fragile]{Dictionary}
\begin{itemize} 
\item d[key] : dict에서 key에 해당되는 value를 반환합니다. 
\item d[key] = val : dict에서 key에 해당되는 value를 val로 업데이트합니다. 
\item d.keys() : dict의 key들을 반환합니다. 
\end{itemize}

\lstinputlisting[style=python, 
                caption={Operations for Mapping Types  (example\_dict2.py)}]{"../../Reference/examples/example_dict2.py"}

\end{frame}

\subsection{Functions}

\begin{frame}[fragile]{Function Definition Syntax}
\begin{lstlisting}[language=Python]
def function_name(function_arguments):
    ...
\end{lstlisting}

\begin{itemize} 
\item def : 함수 정의 키워드입니다. 
\item function : 함수 이름을 나타냅니다. 
\item args : 함수 인자입니다. 아래와 같은 옵션이 있습니다. 
\begin{itemize} 
\item arg 
\item arg\_default : 함수 인자의 기본값을 정해줄 때, =을 이용하여 기본값을 지정해줄 수 있습니다. 
\item *arg\_list : 정해지지 않은 수의 인자를 받고자 할 때, *을 하나 붙여서 들어온 인자를 배열로 받을 수 있습니다. 
\item **arg\_dict : 정해지지 않은 수의 이름이 명시된 인자를 받고자 할 때, *를 두개 붙여서 들어온 인자들을 dict 형태로 받을 수 있습니다. 
\end{itemize} 
\item return None : 함수의 결과값으로 return 뒤의 구문을 반환합니다. 
\end{itemize}
\end{frame}


\begin{frame}[fragile]{실습 : sorting a list}
정수의 리스트가 주어졌을 때, 이를 크기순으로 배열한 리스트를 반환하는 함수 sort\_list를 작성하세요. 
\end{frame}

\begin{frame}[fragile]{Lambda Function}
\lstinputlisting[style=python,  
                caption={Lambda Function Example (example\_lambda1.py)}]{"../../Reference/examples/example_lambda1.py"}

파이썬에서는 익명함수(lambda function)을 지원합니다. 이 때 구문은 다음과 같습니다. 
\begin{itemize} 
\item lambda : 람다함수 키워드. 람다함수 뒤의 구문 중 콜론 전에 있는 구문은 함수의 인자를, 뒤는 반환하는 값을 나타낸다. 
\item x,y,z(func3)/*args(func4) : 람다함수의 인자. 쉼표로 구분되며, 상기된 *args등도 똑같이 사용 가능함을 func4에서 확인할 수 있다. 
\item x+y+z(func3)/sum(args)(func4) : 람다함수의 반환값. 
\end{itemize}
\end{frame}

\begin{frame}[fragile]{Functional Python : Function as a First-Class Citizen}
파이썬에서 함수는 다른 객체와 똑같이 취급됩니다. 즉, 
\begin{itemize} 
\item 다른 함수에 인자로 넘길 수 있으며, \footnote{합성함수와는 다릅니다!}
\item 함수의 반환값이 될 수 있고, 
\item 변수에 저장될 수 있습니다. 
\end{itemize}

이는 자바나 C에서는 없는 특성이기 때문에, 예시를 들어서 살펴보도록 하겠습니다. 
\end{frame}

\begin{frame}{실습 : sorting a list using custom key} 
이번에는 정수의 리스트를, 정수를 4로 나눈 나머지를 기준으로 배열해보겠습니다. 
\end{frame}

% \begin{frame}[fragile]{Map, Reduce, Filter}
% \end{frame}

\subsection{Classes} 

\begin{frame}[fragile]{Class Syntax}

\lstinputlisting[style=python,  
		linerange = {1-15},
                caption={Making a MyString Class  (example\_class2.py)}]{"../../Reference/examples/example_class2.py"}
Reference/examples/example\_class2.py 파일을 참고하세요. 
\end{frame}
\begin{frame}
위 코드에서 파이썬 클래스의 각 요소들을 설명하면 다음과 같습니다. 
\begin{itemize} 
\item class : 클래스를 정의하는 키워드입니다. 이 키워드 뒤의 단어가 클래스의 이름이 됩니다. 여기서는 MyString입니다. 
\item maker = 'Schin' : 클래스의 \textbf{속성}을 지정합니다. 여기서는 class\_name, maker, datum의 3개의 속성이 있습니다. 편의상 datum을 우리가 다루고자 하는 문자열로 생각하겠습니다. 
\item a = MyString : a라는 변수의 값으로 MyString이라는 \textbf{클래스}를 지정합니다. 
\item b = MyString() : b라는 변수의 값으로 MyString이라는 \textbf{인스턴스}를 지정합니다. 
\end{itemize}
\end{frame}

% \begin{frame}[fragile]{Defining and Using Methods}
% \begin{lstlisting}[language=Python]
% a = MyString('123')
% a.
% \end{lstlisting}
% \end{frame}

\begin{frame}[fragile]{Syntactic Sugar : Magic Methods}
2 by 2 행렬 클래스를 만들어 봅시다. 이 때, 클래스의 두 인스턴스를 더하거나 빼는 메소드를 만들어 봅시다. 

파이썬에서는 더하기, 빼기 등의 자주 쓰이는 operation들을 우리가 만든 클래스에서 편하게 커스터마이징하는 방법을 제공합니다. 이를 magic method라고 합니다. 많이 쓰이는 magic method와 대응되는 파이썬 문법은 다음과 같습니다. 

\begin{itemize} 
\item \_\_cmp\_\_ : 대소비교할 때 쓰입니다. 
\item \_\_eq\_\_ : == 에서 쓰입니다.
\item \_\_add\_\_, \_\_sub\_\_, \_\_mul\_\_, \_\_div\_\_ : 사칙연산 
\item \_\_iter\_\_ : 인스턴스가 여러 값을 포함하고 있을 경우, 그 값들을 traverse 할 수 있도록 합니다. a in b에서 쓰입니다. 
\item \_\_str\_\_ : str(a)에서 쓰입니다.
\end{itemize}
\end{frame}

\section{Understanding the Python Interpreter}

\subsection{Concept of Namespace and Substitution} 

\begin{frame}[allowframebreaks]{Variables}
위 질문에 답하기 위해서는 먼저 변수의 bind-bound와 Namespace의 개념에 대해서 알아야 합니다. 일반적으로 변수에는 3가지 종류가 있습니다. 

\begin{itemize} 
\item Bound variable : 어떤 값이나 다른 변수에 의해서 값이 결정되는지 정해진 변수
\item Binding variable : Bound variable의 값을 결정하는 변수 
\item Free variable : Bound되지 않은 변수 
\end{itemize}

Namespace란 variable과 값의 bind-bound를 나타내는 구조입니다. 여기에 없는 variable에 대한 접근은 NameError을 반환합니다. 
\end{frame}

\begin{frame}[fragile]{To Bind a Variable}
파이썬에서 Variable을 bind하기 위해서는 

\begin{itemize} 
\item =을 이용한 assignment 
\item class/function 이름
\item 루프에서 루프 카운터의 bind
\item 함수 argument 
\end{itemize}

등\footnote{이 외에는 try-except나 with 구문이 있습니다.}이 있습니다. 

\end{frame}

\begin{frame}[fragile]{Loading Namespaces}

다른 코드 블럭의 Namespace를 가져오기 위해서는 2 가지 방법이 있습니다. 
\begin{itemize} 
\item .의 이용 : AAA.bb는 AAA 코드 블럭 내에서 bind된 bb를 찾는 구문입니다. 
\item from ... import * : ... 모듈이나 파일 안에서 정의된 모든 name을 bind합니다. 
\item Syntactic Sugar : a.some\_method()는 a의 클래스 정의부 내에서 bind된 some\_method를 찾습니다. 즉, 이는 A.some\_method(a)와 같습니다. 
\end{itemize}
\end{frame}

\begin{frame}[fragile]{Explicit Checking of Namespace : locals()}
locals() 함수를 사용하여 특정 시점에서의 namespace를 체크할 수 있습니다. 어떠한 종류의 디버거던 사용해 보신 분들은 익숙한 개념일 것입니다. 
\end{frame}

\subsection{Interpreting Python Code} 

\begin{frame}[fragile]{Execution Flow of the Interpreter}
파이썬 코드는 인덴트로 구분된 코드 블럭 단위로 실행됩니다. 한 코드 블럭 안에서는 Flow Control 구문이 있을 시 그 구문을 따르고, 그렇지 않으면 라인 하나하나씩 실행됩니다. 

라인 중 =가 있는 라인과 없는 라인은 다르게 작동합니다. =가 있는 라인은, =의 왼쪽에는 variable, 오른쪽에는 expression이 있습니다. expression은 계산 가능한 코드나, 기존에 값이 assign된 변수입니다. 없는 라인은 기존에 assign된 변수의 값을 변화시키거나, 다른 함수를 실행합니다. 

그렇다면, expression의 계산이나 함수의 실행은 어떻게 이루어질까요?
\end{frame}


\begin{frame}[fragile]{Substitution}
파이썬 인터프리터는 단순합니다. 그 시점에서의 Namespace에서 variable을 찾고, 찾지 못할 경우 NameError를 리턴합니다. 만약 찾을 수 있는 경우, 그 코드의 해당 부분을 namespace에서 찾은 값으로 바꿔쓰기하여 계산을 다시 수행합니다. 
\end{frame}

\begin{frame}[fragile]{Example on Sorting}
위에서 custom key를 가진 sorting함수를 다시 살펴보며 어떤 식으로 bind-bound가 이루어지는지 살펴보겠습니다. 
\end{frame}


\end{document}


