% Copyright 2004 by Till Tantau <tantau@users.sourceforge.net>.
%
% In principle, this file can be redistributed and/or modified under
% the terms of the GNU Public License, version 2.
%
% However, this file is supposed to be a template to be modified
% for your own needs. For this reason, if you use this file as a
% template and not specifically distribute it as part of a another
% package/program, I grant the extra permission to freely copy and
% modify this file as you see fit and even to delete this copyright
% notice. 

\documentclass{beamer}

% There are many different themes available for Beamer. A comprehensive
% list with examples is given here:
% http://deic.uab.es/~iblanes/beamer_gallery/index_by_theme.html
% You can uncomment the themes below if you would like to use a different
% one:
%\usetheme{AnnArbor}
%\usetheme{Antibes}
%\usetheme{Bergen}
%\usetheme{Berkeley}
%\usetheme{Berlin}
%\usetheme{Boadilla}
%\usetheme{boxes}
%\usetheme{CambridgeUS}
%\usetheme{Copenhagen}
%\usetheme{Darmstadt}
%\usetheme{default}
%\usetheme{Frankfurt}
%\usetheme{Goettingen}
%\usetheme{Hannover}
%\usetheme{Ilmenau}
%\usetheme{JuanLesPins}
%\usetheme{Luebeck}
\usetheme{Madrid}
%\usetheme{Malmoe}
%\usetheme{Marburg}
%\usetheme{Montpellier}
%\usetheme{PaloAlto}
%\usetheme{Pittsburgh}
%\usetheme{Rochester}
%\usetheme{Singapore}
%\usetheme{Szeged}
%\usetheme{Warsaw}

\usepackage{kotex}
\usepackage{braket}
\usepackage{array}
\usepackage{calc}
\usepackage{datetime}


\graphicspath{{images/}}
\usepackage{listings}

\newcommand\fig[2]{
\begin{figure}[h]
  \centering
  \includegraphics[width = \textwidth]{#1}
  \caption{#2} 
  \label{fig:#1}
\end{figure}
}

\usepackage{kotex} 
\usepackage{hyperref}
\usepackage{listings}
\usepackage{color}

\definecolor{dkgreen}{rgb}{0,0.6,0}
\definecolor{gray}{rgb}{0.5,0.5,0.5}
\definecolor{mauve}{rgb}{0.58,0,0.82}

\definecolor{mygreen}{rgb}{0,0.6,0}
\definecolor{mygray}{rgb}{0.5,0.5,0.5}
\definecolor{mymauve}{rgb}{0.58,0,0.82}

\lstset{ 
  backgroundcolor=\color{white},   % choose the background color; you must add \usepackage{color} or \usepackage{xcolor}; should come as last argument
  basicstyle=\footnotesize,        % the size of the fonts that are used for the code
  breakatwhitespace=false,         % sets if automatic breaks should only happen at whitespace
  breaklines=true,                 % sets automatic line breaking
  captionpos=b,                    % sets the caption-position to bottom
  commentstyle=\color{mygreen},    % comment style
  deletekeywords={...},            % if you want to delete keywords from the given language
  escapeinside={\%*}{*)},          % if you want to add LaTeX within your code
  extendedchars=true,              % lets you use non-ASCII characters; for 8-bits encodings only, does not work with UTF-8
  frame=single,	                   % adds a frame around the code
  keepspaces=true,                 % keeps spaces in text, useful for keeping indentation of code (possibly needs columns=flexible)
  keywordstyle=\color{blue},       % keyword style
  language=Python,                 % the language of the code
  morekeywords={*,...},            % if you want to add more keywords to the set
  numbers=left,                    % where to put the line-numbers; possible values are (none, left, right)
  numbersep=5pt,                   % how far the line-numbers are from the code
  numberstyle=\tiny\color{mygray}, % the style that is used for the line-numbers
  rulecolor=\color{black},         % if not set, the frame-color may be changed on line-breaks within not-black text (e.g. comments (green here))
  showspaces=false,                % show spaces everywhere adding particular underscores; it overrides 'showstringspaces'
  showstringspaces=false,          % underline spaces within strings only
  showtabs=false,                  % show tabs within strings adding particular underscores
  stepnumber=2,                    % the step between two line-numbers. If it's 1, each line will be numbered
  stringstyle=\color{mymauve},     % string literal style
  tabsize=2,	                   % sets default tabsize to 2 spaces
  title=\lstname                   % show the filename of files included with \lstinputlisting; also try caption instead of title
}


\lstdefinestyle{python}{frame=tb,
  language=Python,
  aboveskip=3mm,
  belowskip=3mm,
  showstringspaces=false,
  columns=flexible,
  basicstyle={\small\ttfamily},
  numbers=left,
  numberstyle=\tiny\color{gray},
  keywordstyle=\color{blue},
  commentstyle=\color{dkgreen},
  stringstyle=\color{mauve},
  breaklines=true,
  breakatwhitespace=true,
  tabsize=4
}


\title{Problems on Basic Data Structures}

% A subtitle is optional and this may be deleted
\subtitle{Koscom Algorithm Lecture}

\author{신승우}
% - Give the names in the same order as the appear in the paper.
% - Use the \inst{?} command only if the authors have different
%   affiliation.

% \institute[Universities of Somewhere and Elsewhere] % (optional, but mostly needed)
% {
  % \inst{1}%
  % Department of Computer Science\\
  % University of Somewhere
  % \and
  % \inst{2}%
  % Department of Theoretical Philosophy\\
  % University of Elsewhere}
% - Use the \inst command only if there are several affiliations.
% - Keep it simple, no one is interested in your street address.

% - Either use conference name or its abbreviation.
% - Not really informative to the audience, more for people (including
%   yourself) who are reading the slides online

\subject{Theoretical Computer Science}

% This is only inserted into the PDF information catalog. Can be left
% out. 

% If you have a file called "university-logo-filename.xxx", where xxx
% is a graphic format that can be processed by latex or pdflatex,
% resp., then you can add a logo as follows:

% \pgfdeclareimage[height=0.5cm]{university-logo}{university-logo-filename}
% \logo{\pgfuseimage{university-logo}}

% Delete this, if you do not want the table of contents to pop up at
% the beginning of each section:


% \AtBeginSection[]
% {
  % \begin{frame}[fragile]<beamer>{Outline}
    % \tableofcontents[currentsection,hideallsections]
  % \end{frame}
% }

% Let's get started
\begin{document}

\begin{frame}
  \titlepage
\end{frame}

\begin{frame}{Outline}
  \tableofcontents
  % You might wish to add the option [pausesections]
\end{frame}

% Section and sections will appear in the presentation overview
% and table of contents.

% \begin{frame}[fragile][fragile]{hello.py}
% 이제는 위에서 했던 것과 거의 비슷한 것을 할 텐데, 다만 다른 방식으로 해 볼 것입니다. 화면에서 보이는 대로 따라하세요. 
% \begin{lstlisting}[language=Python]
% print('Hello World!')
% \end{lstlisting}
% \end{frame}


% \begin{frame}{}

% \end{frame}

\section{Evolution Algorithm}

\begin{frame}{Introduction to Genetic Algorithm}

유전 알고리즘은 어떤 문제를 해결하기 위한 가설들을 자연에서의 유전 과정과 흡사한 방법을 이용하여 진화시키고, fitness function을 이용하여 이를 검증, 발전시켜나가는 방법입니다. 대단히 우연성에 강하게 의존할 것 같은 알고리즘이지만, 의외로 많은 곳에서 활발하게 쓰이고 있습니다. 
\end{frame}

\begin{frame}{Process of Genetic Algorithm}
크게 

\begin{itemize} 
\item 초기화 : 가설들을 유전자로 표현하고, 랜덤한 유전자를 생성합니다. 
\item 선택 : 초기화된 유전자들을 \textbf{fitness} 함수를 이용하여 평가하고, 이를 선택한다. 
\item 교차 : 선택된 유전자들을 확률적으로 섞여 변이를 일으켜, 다음 세대의 유전자를 만든다. 
\item 변이 : 만들어진 다음 세대 유전자에 인위적으로 돌연변이를 일으킨다. 대개의 경우 매우 낮은 확률로 변이를 준다. 
\item 대치 : 후대 유전자로 현 유전자 풀을 교체한다. 
\item 반복 : 이 과정을 반복한다. 
\end{itemize}

의 과정을 거친다. 이를 위해서, fitness함수를 적절히 정의하는 것이 필수적이다. 
\end{frame}

\begin{frame}{유전 알고리즘의 예시}
0000-1111 중 가장 큰 이진수를 찾아보는 문제를 풀어보자. 이 때, 유전자는 정수 4개의 리스트로 생각할 수 있다. 여기서 fitness 함수는 1의 갯수로 두고 진화를 시켜본다고 생각하자. 
\end{frame}

\begin{frame}{유전 알고리즘의 예시 : 초기화}
10개의 유전자를 아래와 같이 랜덤하게 생성하였다. 

[0, 0, 1, 0], [0, 1, 0, 0], [1, 0, 1, 0], [1, 1, 0, 1], [0, 0, 0, 0], [0, 0, 0, 0], [1, 0, 0, 0], [1, 0, 1, 1], [0, 0, 1, 1], [0, 0, 0, 1]

\end{frame}
\begin{frame}{유전 알고리즘의 예시 : 선택}
위에서 정한 fitness 함수가 높은 유전자 순으로 배열하면 아래와 같다. 

[1, 1, 0, 1], [1, 0, 1, 1]  : 3
 
[1, 0, 1, 0], [0, 0, 1, 1] : 2

[0, 0, 1, 0], [0, 1, 0, 0], [1, 0, 0, 0], [0, 0, 0, 1] : 1

[0, 0, 0, 0], [0, 0, 0, 0] : 0

그 후, 이 중 상위 몇 개만을 골라\footnote{고르는 방법은 토너먼트 선택, 랭킹 기반 선택 등 다양한 방법이 있다.} 다음 세대를 만드는 데 사용할 것이다. 

\end{frame}
\begin{frame}{유전 알고리즘의 예시 : 교차}

앞에서 골라진 유전자가 다음의 4개라고 하자. 

[1, 1, 0, 1], [1, 0, 1, 1], [1, 0, 1, 0], [0, 0, 1, 1]

이 때, 이를 랜덤하게 교배한다. 여기서는 간단하게 두 유전자가 같은 비트는 그대로 사용하고, 다른 비트는 둘 중 랜덤하게 고르기로 하자. 예를 들어서 아래와 같이 교배할 수 있다. 

[1, 1, 0, 1] + [1, 0, 1, 1] = [1*, 1-, 1+, 1*]

이렇게 교배하여 만든 유전자 중 상위 10개를 고르자. 
\end{frame}
\begin{frame}{유전 알고리즘의 예시 : 변이/대치}
골라진 10개 중 랜덤하게 돌연변이를 줄 수 있다. 돌연변이를 준 후, 다음 세대의 유전자로 기존 유전자를 교체한 후, 이를 반복한다. 
\end{frame}


\begin{frame}{Application of Genetic Algorithm}
\begin{itemize}
\item AutoML : Applying Evolution Algorithm in Deep Learning 
\item Fault-Localization Problem 
\end{itemize}
\end{frame}

\begin{frame}{Spectra-based Fault-Localization Problem}
어떤 프로그램에 대해서 m개의 테스트를 시행하였다. 이 때, 
\begin{itemize} 
\item Spectra : 프로그램의 각 라인에 대해서, $(e_p, e_f, n_p, n_f)$의 튜플을 말한다. 이 때, $e_p, e_f$는 각각 그 라인이 통과한/실패한 테스트에서 실행된 횟수를 말하고, $n_p, n_f$는 각각 그 라인이 통과한/실패한 테스트에서 실행되지 않은 횟수를 말한다. 
\item Fault-Localization : 테스트를 통과하지 못한 것이 어떤 라인에 의한 것인지 찾는 것을 의미한다. 
\end{itemize}

이 때, Spectra를 이용해서 Fault Localization 문제를 푸는 것을 Spectra-based Fault Localization이라고 한다. 
\end{frame}

\begin{frame}{Risk Evaluation Metric}
SBFL은 일반적으로 spectra에 기반한 metric들을 이용하여 각 라인이 얼마나 에러에 기여한지 평가한다. 예컨대, $e_f$가 실패한 테스트의 수와 거의 일치한다면, 그 라인이 에러에 기여했을 확률은 지극히 높을 것이다. 이런 metric을 Risk Evaluation Metric이라고 하며, 좋은 metric을 찾는 것이 SBFL에서 매우 중요하다. 
\end{frame}

\begin{frame}{Applying Genetic Algorithm for Metric Generation}

이제, metric을 tree로 보고, 랜덤하게 유전자를 준비하여 가지치기와 붙이기를 통해서 유전자 교배를 시행, 이를 통해서 metric을 진화시킬 수 있다. 논문 Evolving Human Competitive Spectra-Based Fault Localisation Techniques 에서는 이와 같은 방법을 통해서 인간이 만든 Risk Evaluation Metric과 비슷하거나 더 좋은 성능을 가지는 metric을 만드는 것에 성공하였다. 
\end{frame}

% \section{Hidden Markov Models and Dynamic Programming}


\section{Simulation on 2D Mechanics} 

\begin{frame}{Simulation and Object Oriented Language}
초창기의 객체지향 언어는 시뮬레이션을 위해서 만들어진 경우가 많았습니다. 이는 객체지향이 시뮬레이션을 작성하기에 매우 적합하기 때문입니다. 객체지향으로 시뮬레이션을 작성하기 적합한 이유는, 객체들 간 상호작용 법칙과 초기조건만 지정해 주면 어렵지 않게 객체 간의 상호작용을 일으킬 수 있기 때문입니다. 

가장 대표적인 예시로는 역학 시뮬레이터가 있습니다. 여기서는 2d 공간 내에서 입자간 상호작용을 설정해 주었을 때, 어떤 식으로 시뮬레이션이 작성되는지 살펴보겠습니다. 
\end{frame}

\begin{frame}{Basic Mechanics}

물리적인 시뮬레이션을 만든다면, 객체간의 상호작용 법칙은 뉴턴역학을 따를 것입니다. 따라서 간단하게 뉴턴역학을 살펴보고자 합니다. 

역학에서 풀고자 하는 문제는, 주어진 역장 $\vec{F}$에 대해서 $\vec{r}(t)$를 구하는 것을 목표로 합니다. 이 때, 이를 구하는 방법은 뉴턴의 제 2법칙인 $F = ma$를 이용하여 구할 수 있습니다. 

\end{frame}

\begin{frame}{Ideal Gas Simulation}
PV = nRT의 시뮬레이션을 통한 검증을 해 보겠습니다. 
\end{frame}


\section{Fourier Transform}

\begin{frame}{Motivating Example} 
핸드폰 기지국에서 고객에게 신호를 보내는 상황을 생각해 봅시다. 이 때, 

\begin{itemize} 
\item 모든 고객이 받는 신호는 같습니다. 
\item 그런데, 모든 고객은 거기서 각자 다른 신호를 받아야만 합니다. 
\end{itemize}

어떻게 하면 여러 고객에게 각자에 맞는 신호를 전달할 수 있을까요? 이를 해결하기 위해서 여러 방법이 제시되었는데, 그 중 하나가 Fourier Transform을 이용한 것입니다. 
\end{frame}

\begin{frame}{Motivating Example : Fourier Transform 들여다보기} 

먼저, 삼각함수의 다음 성질들을 살펴보자. 

\begin{itemize}
\item $sin(x+y) = sin(x)cos(y) + sin(y)cos(x)$
\item $cos(x+y) = cos(x)cos(y) - sin(x)sin(y)$
\item $tan(x+y) = \frac{tan(x)tan(y)}{1-tan(x)tan(y)}$
\end{itemize}

위 식을 이용해서, 1) $sin(nx)sin(mx) = \frac{1}{2} (cos(m-n)x - cos(m+n)x)$임을 얻을 수 있다. 

\end{frame}

\begin{frame}{Proof of 1)} 
위 삼각함수 합차공식 중 $cos(x+y) = cos(x)cos(y) - sin(x)sin(y)$ 에서, x 대신 mx, y 대신 nx를 대입하면 다음과 같다. 

$cos((m+n)x) = cos(mx)cos(nx) - sin(nx) sin(mx)$ 

비슷하게, mx, -nx를 각각 대입하면 다음과 같다. 

$cos((m-n)x) = cos(mx)cos(-nx) - sin(mx) sin(-nx)$

여기서 $cos(-x) = cos(x), sin(-x) = -sin(x)$임을 이용하고, 위 두 식을 빼면 위 결과를 얻을 수 있다. 

\end{frame}

\begin{frame}{확장된 Motivation의 예시 : Fourier Transform 들여다보기} 

여기서, 위 1)을 이용하여 2) $\int^{\pi}_{-\pi} sin(nx)sin(mx) dx = C\delta_{nm}$ 임을 보일 수 있다. 여기서 $\delta_{nm}$은 n=m이면 1이고, 아니면 0인 함수이다.

\end{frame}

\begin{frame}{Proof of 2)} 
$\int cos(nx) dx = \frac{1}{n} sin(nx) + D$ 이다. 따라서 

$\int \frac{1}{2} (cos(m-n)x - cos(m+n)x) dx = \frac{1}{2(m-n)} sin((m-n)x) - \frac{1}{2(m+n)} sin((m+n)x)$ 

이다. 여기서 각각 $\pi, -\pi$를 넣어 계산하면 모든 항이 0이 되므로 0이다. 만약 m=n이라면, 

$\int\frac{1}{2} (cos(0) - cos(2nx) dx = \frac{x}{2}$ 이므로 $\pi$가 된다. 

\end{frame}

\begin{frame}{확장된 Motivation의 예시 : 유사벡터} 

이제, 다음과 같은 집합 2개와 급수를 생각해 보자. 

\begin{itemize} 
\item $[-\pi, \pi]$를 정의역으로 가지는 모든 연속함수의 집합 F
\item $S = \{sin(nx)|n = 1,2,....\} \cup {1}$ 
\item $\sum^{\inf}_{n=0} [\int^{\pi}_{-\pi} f(x) sin(nx) dx] sin (nx) $
\end{itemize}

여기서, 위 두 슬라이드에서의 결과($\int^{\pi}_{-\pi} sin(nx)sin(mx) dx = C\delta_{nm}$)를 이용하면 S의 원소들을 이용해서 F에서 일종의 좌표계를 얻을 수 있음을 알 수 있다. 이 좌표계는 $f(x) \rightarrow (\int^{\pi}_{-\pi} f(x) sin(nx) dx), n=0,1,...$ 으로 주어진다. 
\end{frame}


\begin{frame}{확장된 Motivation의 예시 : 기지국에서 신호 보내기} 

따라서, 다음과 같이 기지국의 신호를 생각할 수 있다. 

\begin{itemize} 
\item 기지국에서 신호를 n명에게 보낸다고 할 때, 
\begin{itemize}
\item n명 각각은 0,1로 된 stream을 받을 수 있어야 한다. 
\item 따라서, 2n개의 sin 기저를 준비해서 각각 유저에게 배포하고, 
\item 유저는 본인이 가진 벡터를 이용하여 신호를 '내적' 하여, 남은 결과값만을 분석한다. 
\end{itemize}
\end{itemize}

\end{frame}


\begin{frame}{Implementation}

Coding! 

\end{frame}

\end{document}


