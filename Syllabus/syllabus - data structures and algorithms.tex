\documentclass[11pt, a4paper]{article}
%\usepackage{geometry}
\usepackage[inner=2.5cm,outer=2.5cm,top=2.5cm,bottom=2.5cm, textwidth=5cm]{geometry}
\pagestyle{empty}
\usepackage{graphicx}
\usepackage{fancyhdr, lastpage, bbding, pmboxdraw}
\usepackage[usenames,dvipsnames]{color}
\definecolor{darkblue}{rgb}{0,0,.6}
\definecolor{darkred}{rgb}{.7,0,0}
\definecolor{darkgreen}{rgb}{0,.6,0}
\definecolor{red}{rgb}{.98,0,0}
\usepackage[colorlinks,pagebackref,pdfusetitle,urlcolor=darkblue,citecolor=darkblue,linkcolor=darkred,bookmarksnumbered,plainpages=false]{hyperref}
\renewcommand{\thefootnote}{\fnsymbol{footnote}}



\pagestyle{fancyplain}
\fancyhf{}
\lhead{ \fancyplain{}{MAS101} }
%\chead{ \fancyplain{}{} }
\rhead{ \fancyplain{}{\today} }
%\rfoot{\fancyplain{}{page \thepage\ of \pageref{LastPage}}}
\fancyfoot[RO, LE] {page \thepage\ of \pageref{LastPage} }
\thispagestyle{plain}

%%%%%%%%%%%% LISTING %%%
\usepackage{listings}
\usepackage{caption}
\DeclareCaptionFont{white}{\color{white}}
\DeclareCaptionFormat{listing}{\colorbox{gray}{\parbox{\textwidth}{#1#2#3}}}
\captionsetup[lstlisting]{format=listing,labelfont=white,textfont=white}
\usepackage{verbatim} % used to display code
\usepackage{fancyvrb}
\usepackage{acronym}
\usepackage{kotex}
\usepackage{amsthm}
\VerbatimFootnotes % Required, otherwise verbatim does not work in footnotes!



\definecolor{OliveGreen}{cmyk}{0.64,0,0.95,0.40}
\definecolor{CadetBlue}{cmyk}{0.62,0.57,0.23,0}
\definecolor{lightlightgray}{gray}{0.93}



\lstset{
%language=bash,                          % Code langugage
basicstyle=\ttfamily,                   % Code font, Examples: \footnotesize, \ttfamily
keywordstyle=\color{OliveGreen},        % Keywords font ('*' = uppercase)
commentstyle=\color{gray},              % Comments font
numbers=left,                           % Line nums position
numberstyle=\tiny,                      % Line\itemnumbers fonts
stepnumber=1,                           % Step between two line\itemnumbers
numbersep=5pt,                          % How far are line\itemnumbers from code
backgroundcolor=\color{lightlightgray}, % Choose background color
frame=none,                             % A frame around the code
tabsize=2,                              % Default tab size
captionpos=t,                           % Caption\itemposition = bottom
breaklines=true,                        % Automatic line breaking?
breakatwhitespace=false,                % Automatic breaks only at whitespace?
showspaces=false,                       % Dont make spaces visible
showtabs=false,                         % Dont make tabls visible
columns=flexible,                       % Column format
morekeywords={__global__, __device__},  % CUDA specific keywords
}

%%%%%%%%%%%%%%%%%%%%%%%%%%%%%%%%%%%%
\begin{document}
\begin{center}
{\Large \textsc{Review on Algorithms and Data Structures}}
\end{center}
\begin{center}
\today
\end{center}
%\date{September 26, 2014}

\begin{center}
\rule{6in}{0.4pt}
\begin{minipage}[t]{.75\textwidth}
\begin{tabular}{llcccll}
\textbf{Lecturer:} & Seungwoo Schin & & &  & \textbf{Time:} & TBA \\
\textbf{Email:} &  \href{mailto:principia_12@kaist.ac.kr}{principia\_12@kaist.ac.kr} & & & & \textbf{Place:} & TBA
\end{tabular}
\end{minipage}
\rule{6in}{0.4pt}
\end{center}
\vspace{.5cm}
\setlength{\unitlength}{1in}
\renewcommand{\arraystretch}{2}


\noindent\textbf{Study Objectives} 데이터구조와 알고리즘을 복습하고 구현하며, 이를 이용하여 ICPC 등의 기출문제를 포함한 종합적인 문제를 풀어본다. 


\vskip.15in
\noindent\textbf{Target Audience} 
\begin{itemize}
\item 대학교 전공자 수준의 알고리즘 지식을 복습하고 싶으신 분 
\end{itemize}


\vskip.15in
\noindent\textbf{Things to Learn} 
\begin{itemize}
\item 데이터구조의 개념과 구현법
\item 그래프, 트리 등 데이터구조의 operation들의 이해와 분석  
\item ICPC 등 경시대회 문제 및 복합적인 문제 풀이 
\end{itemize}

\vskip.15in
\noindent\textbf{Policy} 
\begin{itemize} 
\item 앞에서 구현한 코드를 이용하여 문제풀이에서 활용할 예정입니다. 
\item 실습 코드는 개설될 Github Repository에 커밋하여 공유하게 할 예정입니다. 
\item 실습을 하는 것이 의무는 아니지만, 가능하면 실습 문제를 해결해볼 것을 권장드립니다. 
\end{itemize}


\vskip.15in
\noindent\textbf{Prerequisite} : 프로그래밍 경험과 고등학교 수준의 수학 지식을 가지고 있어야 합니다. 또, 가능하면 git을 사용할 줄 알면 좋습니다. 수업 전에 파이썬을 설치하고 오시기를 권장드립니다. 

\newpage

\noindent \textbf{Tentative Course Outline}
\begin{center} 
\begin{flushleft}
\begin{itemize}

\item Week 0(before class) : Environment Settings

\begin{itemize}
\item Python 개발 환경 설치 
\item Github에서 Template 코드 클론받고 버젼관리 설치 
\end{itemize}

\item Day 1 : 파이썬 기초 및 시간복잡도 / ADT와 데이터구조

\begin{itemize} 
\item 파이썬 문법 
\begin{itemize}
\item Hello World! 
\item 파이썬 자료형 : numeric, set, iterables, map
\item 파이썬 기본 문법 : loops, conditionals, functions, class
\item Argument Passing in Python 
\end{itemize} 
\item 복잡도 
\begin{itemize} 
\item Big O notation
\item Computation Model : Turing Machine
\item Computational Complexity Classes 
\end{itemize} 
\item 데이터구조와 ADT 
\begin{itemize} 
\item 데이터구조 vs ADT 
\item Union-find를 통한 ADT와 데이터구조의 차이 예제 
\end{itemize}
\item 리스트 ADT의 구현 
\item 큐와 스택 ADT의 구현 
\begin{itemize}
\item queue/stack adt
\item  linked list를 이용한 스택과 큐의 구현 
\item  priority queue 
\end{itemize} 
\end{itemize}


\item Day 2 : 데이터구조 더 살펴보기 / 알고리즘 살펴보기 

\begin{itemize} 

\item 트리 
\begin{itemize} 
 \item tree adt
\item binary tree : implementation 
\item balanced binary tree : implementation 
\end{itemize} 
\item 그래프 
\begin{itemize} 
\item graph adt 
\item graph implementation 
\item dfs/bfs 
\end{itemize}

\item Warm Up : Sorting 
\begin{itemize} 
\item $n^2$ sort : insertion sort / bubble sort 
\item $n log n$ sort : quicksort / merge sort 
\item optimizing/analyzing sort 
\end{itemize}

\item Dynamic Programming 
\begin{itemize} 
\item LCS/LIS
\item edit distance
\item subset sum problem 
\end{itemize}

\end{itemize}



\item Day 3 : 다양한 그래프 알고리즘 / 문제 풀이 

\begin{itemize} 
\item 그래프 알고리즘 
\begin{itemize}
\item Graph Traversal : BFS / DFS / Topological Sort
\item Shortest Path to all nodes from Source : Dijkstra / Bellman Ford 
\item Shortest Path between all nodes : Floyd Warshall / Johnson
\item Minimum Spanning Tree : Prim / Kruskal 
\end{itemize} 
\item 경시대회 기출문제 풀이  
\begin{itemize} 
\item Recursion  
\item Dynamic Programming : Using nd array / tree / graph
\end{itemize} 
\end{itemize}
\item More Problems 
\begin{itemize} 
\item Sequence Labeling : Viterbi Algorithm 
\item Parsing : Formal Grammar, Regular Expression
\end{itemize}

\end{itemize}
\end{flushleft}
\end{center}





%%%%%% THE END 
\end{document} 